\documentclass[twoside,twocolumn,a4paper,10pt]{memoir}
\usepackage[utf8]{inputenc}
\usepackage{babel}
\usepackage{fontenc}
\usepackage{graphicx}
\usepackage{scripture}

\begin{document}
	
	\begin{center} \large \textbf{PREFACE}\normalsize\end{center}
	
	This excerpt of Matthew is translated from King James Version by Fluxus Cantonia Art Collective. It serves two purposes:
	\begin{enumerate}
		\item as a demonstration of 
		the newest Cantonese romanization from Fluxus Cantonia; and:
		\item as the beginning of a translation attempt of the Bible to modern Cantonese.
	\end{enumerate}
	
	The transliteration of certain names (of people and places) are slightly different from common ``Chinese'' transliteration, since Cantonese Bible being in Cantonese should not subject itself to the restriction of Mandarin phonology as previous attempts tend to do, e.g. the name ``Jacob'' in its original form starts with a ``y'' sound; thus in this version it is translated as ``Y\v{a}a-gok'' instead of the more common ``Ng\v{a}a-gok''.
	
	The translator hereby declare all content of this document to be in public domain; that is, the translator denounce any associating copyright, and anyone can freely use, copy and redistribute this document or the content of this document without any requirements.
	
	\begin{flushright}
		\textit{2024.5.23}
	\end{flushright}
	
	\begin{center} \large \textbf{ON THE ROMANIZATION USED} \normalsize \end{center}
	The following is a quick summary of the difference between Jyutping and the romanization used in this text. Note that this section assumes the reader is familiar with Jyutping.
	
	\begin{itemize}
		\item ``eo'' and ``oe'' are now ``ø'' and ``œ''.
		\item All ``j''s are now ``y''s.
		\item ``Yu'' before ``z'' ``c'' and ``j'' are written as ``u''; and also, ``z'' and ``c'' before ``yu'' are written as ``j'' and ``ch'' (e.g. it's ``jun'' but not ``zyun'', ``chun'' but not ``cyun'', ``yut'' but not ``jyut'' or ``yyut'').
		\item ``Zoe'' and ``coe'' are now ``joe'' and ``choe' (which are in turn ``jœ'' and ``chœ'').
		\item Tones are now diacritics instead of numbers, so it's \Large sî  sí si sï sǐ sĩ \normalsize instead of \Large si1 si2 si3 si4 si5 si6 \normalsize.
	\end{itemize}

	
	
	\vskip 36pt
	\begin{center} \large \textbf{M\v{A}A-TAAI FÛK-YÂM} \normalsize \end{center}
	\begin{scripture}
		\ch{1}\noindent
		\vs{1}\underline{Aa-baak-lâa-häam} ge hãu-dõi, dãai-wãi ge zí-syûn Yë-sôu Gêi-dûk ge gâa póu yü hãa.
		
		\vs{2}\underline{Aa-baak-lâa-häam} sâang \underline{Y\v{i}-saat}, \underline{Y\v{i}-saat} sâang \underline{Y\v{a}a-gok}; \underline{Y\v{a}a-gok} sâang zó \underline{Yäu-dãai} tüng k\v{ø}i ge hîng dãi; \vs{3}\underline{Yäu-dãai} tüng \underline{Tâa-m\v{a}a} sâang zó \underline{Faat-lẽi-sî} tüng \underline{Saat-lâa}; \underline{Faat-lẽi-sî} sâang \underline{Yî-sî-lø̈n}; \underline{Yî-sî-lø̈n} sâang \underline{Aa-läm};
		\vs{4}\underline{Aa-läm} sâang \underline{Aa-m\v{a}i-näa-daap}; \underline{Aa-m\v{a}i-näa-daap} sâang \underline{Näa-sø̃n}; \underline{Näa-sø̃n} sâang \underline{Saat-y\v{i}-mün};
		\vs{5}\underline{Saat-y\v{i}-mün} tüng \underline{Lâa-hãp} sâang zó \underline{Bô-zî}, \underline{Bô-zî} tüng \underline{Lõu-dâk} sâang \underline{Ô-bui-dâk}; \underline{Ô-bui-dâk} sâang \underline{Yë-sî};
		\vs{6}\underline{Yë-sî} sâang zó \underline{Dãai-wãi} wöng, yï \underline{Dãai-wãi} tüng yün bún yîng gôi hãi \underline{Wû-l\v{e}i-aa-sî} ge l\v{o}u pö sâang zó \underline{Só-lö-mün};
		
		\vs{7}\underline{Só-lö-mün} sâang \underline{Lö-bô-am}; \underline{Lö-bô-am} sâang \underline{Aa-béi-aa}, \underline{Aa-béi-aa} sâang \underline{Aa-sâa};
		\vs{8}\underline{Aa-sâa} sâang \underline{Yœk-sâa-faat}; \underline{Yœk-sâa-faat} sâang \underline{Yœk-läm}; \underline{Yœk-läm} sâang \underline{Ô-zî-aa-sî};
		\vs{9}\underline{Ô-zî-aa-sî} sâang \underline{Yœk-täam}; \underline{Yœk-täam} sâang \underline{Aa-hâa-zî}; \underline{Aa-hâa-zî} sâang \underline{Yî-zî-këi-aa-sî};
		\vs{10}\underline{Yî-zî-këi-aa-sî} sâang \underline{M\v{a}a-näa-sî}; \underline{M\v{a}a-näa-sî} sâang \underline{Aa-mün}; \underline{Aa-mün} sâang \underline{Yœk-sî-aa-sî};
		\vs{11}\underline{Yœk-sî-aa-sî} hái bei yän daai dou \underline{Bâa-béi-lø̈n} gó zãn sâang zó \underline{Yë-ô-nëi-aa-sî} tüng k\v{\o{}}i ge hîng dãi.
		
		\vs{12}Cîn dou \underline{Bâa-béi-lø̈n} zî hãu, \underline{Yë-ô-nëi-aa-sî} sâang \underline{Saat-lâa-tit-y\v{i}}; \underline{Saat-lâa-tit-y\v{i}} sâang \underline{Zo-lö-bâa-bui-y\v{i}};
		\vs{13}\underline{Zo-lö-bâa-bui-y\v{i}} sâang \underline{Aa-béi-wû-dâk}; \underline{Aa-béi-wû-dâk} sâang \underline{Yî-l\v{e}i-aa-gîm}; \underline{Yî-l\v{e}i-aa-gîm} sâang \underline{Aa-zo-y\v{i}}
		\vs{14}\underline{Aa-zo-y\v{i}} sâang \underline{Sâa-dõk}; \underline{Sâa-dõk} sâang \underline{Aa-hêi-m\v{o}u}; \underline{Aa-hêi-m\v{o}u} sâang \underline{Y\v{i}-l\~{\o{}}t};
		\vs{15}\underline{Y\v{i}-l\~{\o{}}t} sâang \underline{Yî-l\v{e}i-aa-sâa}; \underline{Yî-l\v{e}i-aa-sâa} sâang \underline{M\v{a}a-daan}; \underline{M\v{a}a-daan} sâang \underline{Y\v{a}a-gok};
		\vs{16}\underline{Y\v{a}a-gok} sâang \underline{M\v{a}a-lẽi-aa} ge l\v{o}u gûng \underline{Yœk-sât}; bẽi cîng wäi Gêi-dûk ge Yë-sôu zîk hãi \underline{M\v{a}a-lẽi-aa} sâang ge.
		
		\vs{17}Yü s\~{œ}ng só sø̃t, cüng \underline{Aa-baak-lâa-häam} dou \underline{Dãai-wãi} gwo zó sãp-sei dõi, cüng \underline{Dãai-wãi} høi dou gwo \underline{Bâa-béi-lø̈n} ge sï hãu gwo zó sãp-sei dõi, cüng dou zó \underline{Bâa-béi-lø̈n} høi dou Yë-sôu cø̂t sai yãu gwo zó sãp-sei dõi .
		
		\vs{18}Yë-sôu Gêi-dûk cø̂t sai hãi gám ge yât wüi sĩ: k\v{ø}i aa-mâa \underline{M\v{a}a-lẽi-aa} dông sï y\v{i} gîng hǿi zó bei \underline{Yœk-sât}, dãan hãi hái kø̌i zũng mẽi gwo mün ge sï hãu, zãu yân sing lëng yï dãai zó t\v{o}u.
		\vs{19}kø̌i l\v{o}u gûng \underline{Yœk-sât} zok wäi yât go yĩ yän m̈ yũn yĩ dông gâai sâu-yũk kø̌i , yû-sĩ zãu nám jũ sî dái hãa yâu zó kø̌i.
		\vs{20}Dãan hãi hái kø̌i nám gán nî-dî sĩ ge sï hãu, \name{s\~{œ}ng-dai} ge sí-zé hái kø̌i ge mũng yãp bîn cø̂t yĩn, tüng kø̌i góng: \underline{Dãai-wãi} ge hãu dõi \underline{Yœk-sât}, m̈ hóu gêng cǿi \underline{M\v{a}a-lẽi-aa} gwo mün; kø̌i t\v{o}u yãp bîn gó go hái sing lëng ge zái.
		\vs{21}kø̌i wũi sâang yât go näam zái; n\v{e}i yiu tüng go zái héi mëng giu Yë-sôu; kø̌i wũi cüng kø̌i ge yän män ge zø̃i-ok zî zûng jœ̂ng kø̌i dẽi gáai gau cø̂t läi.
		\vs{22}Nî-dî sĩ hãi wãi zó yiu sïng chün \name{jú} tûng gwo sîn-zî ge háu góng ge syũt wãa:
		\vs{23}Y\v{a}u go chủ nø̉i wũi wäai yãn; kø̌i wũi sâang yât go näam zái, yï kø̌i-dẽi wũi giu kø̌i go mëng \underline{Y\v{i}-m\v{a}a-nõi-lẽi} (zîk hãi "\name{sœ̂ng dai} tüng ng\v{o} dẽi tüng zõi" ge yi sî).
		\vs{24}\underline{Yœk-sât} séng zó zî hãu, zãu on ziu \name{s\~{œ}ng dai} ge sí-zé ge fân-fũ cǿi zó kø̌i l\v{o}u pö gwo mün,
		\vs{25}dãan hãi m\v{o}u tüng kø̌i tüng föng; zĩk dou hãu m\v{e}i kø̌i sâang zó kø̌i ge jœ́ng-zí zî hãu, tüng kø̌i cǿi mëng giu Yë-sôu .
		
		
		
		\ch{2}\noindent
		\vs{1}Yë-sôu hái \underline{Hêi-l\~{\o{}}t} zõu gán gwok-wöng ge sï-hãu hái
		\underline{Yäu-dãai} zî dẽi \underline{Bât-lẽi-häam} c\^{\o{}}t sai; y\v{a}u géi go bok-sĩ
		cüng dûng-mĩn läi dou \underline{Yë-lõu-saat-läm} \vs{2}mãn wãa, gó-go c\^{\o{}}t sai yiu läi zõu \underline{Yäu-taai yän} ge wöng ge
		gó-wãi hái bîn nê? Ng\v{o} dẽi hái dûng-mĩn ge sï-hãu tái dou
		k\v{\o{}}i go lâp sêng, só y\v{i} dãk dâng gwo läi baai k\v{\o{}}i. \vs{3}\underline{Hêi-l\~{\o{}}t} wöng têng dou \added{nî-dî sĩ} zî-hãu go sâm zãu hóu
		\"{m} ôn lõk; sïng-go \underline{Yë-lõu-saat-läm} sëng ge baak-sing dôu tüng
		k\v{o}i yât y\~{œ}ng. \vs{4}K\v{\o{}}i zĩu cäi saai chün-bõu ge zai-sî-jóeng tüng mäai
		män-gâan ge gaau-faat jûn-gâa, mãn k\v{\o{}}i dẽi Gêi-dûk gau gíng
		hái bîn-bĩn c\^{\o{}}t sai. \vs{5}K\v{\o{}}i-dẽi tüng k\v{\o{}}i wãa hái \underline{Yäu-dãai} ge \underline{Bât-lẽi-häam}, yân wäi y\v{a}u sîn-zî gám sé wãa:
		
		\vs{6}\begin{poetry}
			\underline{Bât-lẽi-häam} aa, n\v{e}i hái \underline{Yäu-dãai} zî dẽi l\v{\o{}}i-bĩn,
			N\v{e}i hái \underline{Yäu-dãai} zî dẽi l\v{\o{}}i-bĩn bing \"{m} hãi z\o{}i \"{m} gán yiu	gó go;
			
			Yân wãi y\v{a}u yât go túng zĩ zé wũi cüng n\v{e}i gó dõu c\^{\o{}}t läi,
			K\v{\o{}}i wũi túng zĩ ng\v{o} \underline{Y\v{i}-sîk-lĩ} ge zí män.
		\end{poetry}
		
		\vs{7}Zî hãu \underline{Hêi-l\~{\o{}}t} sî dái hãa giu zó bâan bok-sĩ läi,chöeng-sai mãn k\v{\o{}}i dẽi gó lâp sêng hãi hái mê sï hãu c\^{\o{}}t yĩn ge, \vs{8}yïn zî hãu k\v{\o{}}i zãu paai k\v{\o{}}i-dẽi h\o{}i \underline{Bât lẽi häam} dõu, wãa, n\v{e}i dẽi zí-sai dî h\o{}i wán gó-go sai lõu zái, wán dou zî hãu tüng ng\v{o} góng, gám ng\v{o} sîn hóu dôu gwo h\o{}i baai k\v{\o{}}i.  
		\vs{9}K\v{\o{}}i-dẽi têng yün gwok-wöng ge wãa zãu c\^{\o{}}t-faat la; nî-go sï-hãu, k\v{\o{}}i-dẽi hái dûng mĩn ge sï-hãu gin dou ge gó lâp sêng häang hái k\v{\o{}}i-dẽi cïn bĩn, yât zĩk häang dou sai lõu zái gó bîn sîn tïng hái zîng s\v{o}eng fông.
		\vs{10}K\v{\o{}}i-dẽi gin dou lâp sêng, sâm l\v{\o{}}i dãk bĩt gôu-hing.
		\vs{11}Yïn zî hãu k\v{\o{}}i-dẽi yãp dou gâan fóng gin dou go sai lõu zái tüng k\v{\o{}}i aa-mâa \underline{M\v{a}a-lẽi-aa}, yû sĩ zãu fũk dâi baai k\v{\o{}}i; k\v{\o{}}i dẽi dáa hôi k\v{\o{}}i-dẽi daai läi ge bóu-s\^{œ}ng, cïng s\~{œ}ng wöng-gâm, yủ-h\^{œ}ng tüng mũt-y\~{œ}k zok wäi l\v{a}i-mãt.
		\vs{12}Hãu-löi yân wãi \name{S\~{œ}ng-dai} hái k\v{\o{}}i-dẽi ge mũng l\v{\o{}}i-bĩn gíng-gou k\v{\o{}}i-dẽi \"{m} hóu zoi fâan fâan \underline{Hêi-l\~{\o{}}t} gó dõu, só y\v{i} k\v{\o{}}i-dẽi fâan k\v{\o{}}i dẽi zĩ géi dẽi täu ge sï hãu häang zó lĩng ngõi yât tïu lõu.
		
		\vs{13}K\v{\o{}}i-dẽi záu zó zî hãu, \name{jú} ge sí-zé hái \underline{Y\oe{}k-sât} ge mũng yãp bîn c\^{\o{}}t yĩn; tüng k\v{\o{}}i wãa, héi sân lâa! Läa läa läm daai mäai go sai lõu tüng k\v{\o{}}i aa-mâa h\o{}i ôi-kãp, hái gó bîn jũ ju dáng ng\v{o} zoi tûng zî n\v{e}i; \underline{Hêi-l\~{\o{}}t} yiu wán go sai lõu zái c\^{\o{}}t läi saat la.
		\vs{14}Yû-sĩ k\v{\o{}}i síng zó zî hãu zãu tüng go sai lõu tüng k\v{\o{}}i aa-mâa h\o{}i zó Ôi-kãp,
		\vs{15}hái gó bîn jũ dou \underline{Hêi-l\~{\o{}}t} séi zó wäi zí. Nî-dî sĩ hãi wäi zó yiu sïng chün \name{jú} tûng gwo sîn-zî góng ge wãa: Ng\v{o} wũi cüng Ôi-kãp giu Ng\v{o} ge zái c\^{\o{}}t läi.
		\vs{16}\underline{Hêi-l\~{\o{}}t} faat yĩn zĩ géi bei gó bâan bok-sĩ wán zó bãn zî hãu zãu faat saai lãan zâa, paai yän h\o{}i \underline{Bât-lẽi-häam} tüng zâu bîn ge dẽi-k\^{\o{}}i, on ziu k\v{\o{}}i cüng bok-sĩ têng fâan läi ge sï gâan j\^{œ}ng l\v{o}eng s\o{}i y\v{i} hãa ge sai lõu chün-bõu saat saai; 
		\vs{17}gám y\~{œ}ng zãu sïng chün zó sîn-zî \underline{Yë-lẽi-m\v{a}i} góng ge syũt-wãa:
		\vs{18}\begin{poetry}
			Cüng \underline{Lâa-m\v{a}a} sïng chün löi yât báa sêng,
			Bêi tung, câu yâp, tüng ôi dõu;
			
			\underline{Lâa-hit} wãi k\v{\o{}}i ge zái-n\v{\o{}}i yï hûk;
			Möu yän hó y\v{i} ôn wai dou k\v{\o{}}i, yân wäi k\v{\o{}}i ge zái-n\v{\o{}}i chün-bõu dou \"{m} hái dõu la.
		\end{poetry}
		
		\vs{19}\underline{Hêi-l\~{\o{}}t} séi zó zî hãu, \name{s\~{œ}ng-dai} ge sí-zé hái \underline{Yoek-sât} ge mũng yãp bîn c\^{\o{}}t yĩn; 
		\vs{20}tüng k\v{\o{}}i wãa, héi sân lâa! Daai mäai go sai lõu tüng k\v{\o{}}i aa-mâa fâan \underline{Y\v{i}-sîk-lĩt} lâa, yiu saat n\v{e}i go zái ge yän y\v{i} gîng séi zó la.
		\vs{21}Yû-sĩ k\v{\o{}}i zãu héi sân daai go sai lõu tüng k\v{\o{}}i aa-mâa fâan zó \underline{Y\v{i}-sîk-lĩt};
		\vs{22}dãan hãi k\v{\o{}}i têng dou \underline{Aa-hêi-läu-sî} gai sïng zó k\v{\o{}}i l\v{o}u dãu \underline{Hêi-l\~{\o{}}t} go wãi zõu zó \underline{Yäu-dãai} zî dẽi ge gwok wöng ge sï hãu, hôi cí gok dâk gêng yï \"{m} gám fâan h\o{}i; yï-cé \name{s\~{œ}ng-dai} hái k\v{\o{}}i ge mũng yãp bîn dôu gíng gou gwo k\v{\o{}}i, yû-sĩ k\v{\o{}}i zãu jun sân h\o{}i zó \underline{Gâa-lẽi-lẽi};
		\vs{23}K\v{\o{}}i hái yât go giu \underline{Näa-saat-lãak} ge sïng dõu jũ dâi; gám y\~{œ}ng zãu sïng chün zó sîn-zî góng ge <<K\v{\o{}}i wũi bei yän cîng wäi hãi \underline{Näa-saat-lãak} yän>> nî gĩn sĩ.
		
		
		
		
	\end{scripture}
\end{document}
