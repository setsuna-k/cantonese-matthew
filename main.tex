\documentclass[twoside,twocolumn,a4paper,10pt]{memoir}
\usepackage[utf8]{inputenc}
\usepackage{babel}
\usepackage{fontenc}
\usepackage{graphicx}
\usepackage{scripture}

\begin{document}
	
	\begin{center} \large \textbf{PREFACE}\normalsize\end{center}
	
	This version of Matthew is translated from King James Version by Fluxus Cantonia Art Collective. It serves two purposes:
	\begin{enumerate}
		\item as a demonstration of 
		the newest Cantonese romanization; and:
		\item as the beginning of a translation attempt of the Bible to modern Cantonese.
	\end{enumerate}
	
	The transliteration of certain names (of people and places) are slightly different from common ``Chinese'' transliteration, since Cantonese Bible being in Cantonese should not subject itself to the restriction of Mandarin phonology as previous attempts tend to do.
	
	The translator hereby declare all content of this document to be in public domain; that is, the translator denounce any associating copyright, and anyone can freely use, copy and redistribute this document or the content of this document without any requirements.
	
	\begin{flushright}
		\textit{2023.5.23}
	\end{flushright}
	
	\vskip 36pt
	\begin{center} \large \textbf{MẢA-TAAI FÛK-YÂM} \normalsize \end{center}
\begin{scripture}
	\ch{1}\noindent
	\vs{1}\underline{Aa-baak-lâa-häam} ge hãu-dõi, dãai-wãi ge zí-syûn Yë-sôu Gêi-dûk ge gâa póu yü hãa.
	
	\vs{2}\underline{Aa-baak-lâa-häam} sâang \underline{Yỉ-saat}, \underline{Yỉ-saat} sâang \underline{Yảa-gok}; \underline{Yảa-gok} sâang zó \underline{Yäu-dãai} tüng kẻoi ge hîng dãi; \vs{3}\underline{Yäu-dãai} tüng \underline{Tâa-mảa} sâang zó \underline{Faat-lẽi-sî} tüng \underline{Saat-lâa}; \underline{Faat-lẽi-sî} sâang \underline{Yî-sî-lëon}; \underline{Yî-sî-lëon} sâang \underline{Aa-läm};
	\vs{4}\underline{Aa-läm} sâang \underline{Aa-mải-näa-daap}; \underline{Aa-mải-näa-daap} sâang \underline{Näa-sẽon}; \underline{Näa-sẽon} sâang \underline{Saat-yỉ-mün};
	\vs{5}\underline{Saat-yỉ-mün} tüng \underline{Lâa-hãp} sâang zó \underline{Bô-zî}, \underline{Bô-zî} tüng \underline{Lõu-dâk} sâang \underline{Ô-bui-dâk}; \underline{Ô-bui-dâk} sâang \underline{Yë-sî};
	\vs{6}\underline{Yë-sî} sâang zó \underline{Dãai-wãi} wöng, yï \underline{Dãai-wãi} tüng yün bún yîng gôi hãi \underline{Wû-lẻi-aa-sî} ge lỏu pö sâang zó \underline{Só-lö-mün};
	
	\vs{7}\underline{Só-lö-mün} sâang \underline{Lö-bô-am}; \underline{Lö-bô-am} sâang \underline{Aa-béi-aa}, \underline{Aa-béi-aa} sâang \underline{Aa-sâa};
	\vs{8}\underline{Aa-sâa} sâang \underline{Yoek-sâa-faat}; \underline{Yoek-sâa-faat} sâang \underline{Yoek-läm}; \underline{Yoek-läm} sâang \underline{Ô-zî-aa-sî};
	\vs{9}\underline{Ô-zî-aa-sî} sâang \underline{Yoek-täam}; \underline{Yoek-täam} sâang \underline{Aa-hâa-zî}; \underline{Aa-hâa-zî} sâang \underline{Yî-zî-këi-aa-sî};
	\vs{10}\underline{Yî-zî-këi-aa-sî} sâang \underline{Mảa-näa-sî}; \underline{Mảa-näa-sî} sâang \underline{Aa-mün}; \underline{Aa-mün} sâang \underline{Yoek-sî-aa-sî};
	\vs{11}\underline{Yoek-sî-aa-sî} hái bei yän daai dou \underline{Bâa-béi-lëon} gó zãn sâang zó \underline{Yë-ô-nëi-aa-sî} tüng kẻoi ge hîng dãi.
	
	\vs{12}Cîn dou \underline{Bâa-béi-lëon} zî hãu, \underline{Yë-ô-nëi-aa-sî} sâang \underline{Saat-lâa-tit-yỉ}; \underline{Saat-lâa-tit-yỉ} sâang \underline{Zo-lö-bâa-bui-yỉ};
	\vs{13}\underline{Zo-lö-bâa-bui-yỉ} sâang \underline{Aa-béi-wû-dâk}; \underline{Aa-béi-wû-dâk} sâang \underline{Yî-lẻi-aa-gîm}; \underline{Yî-lẻi-aa-gîm} sâang \underline{Aa-zo-yỉ}
	\vs{14}\underline{Aa-zo-yỉ} sâang \underline{Sâa-dõk}; \underline{Sâa-dõk} sâang \underline{Aa-hêi-mỏu}; \underline{Aa-hêi-mỏu} sâang \underline{Yỉ-lẽot};
	\vs{15}\underline{Yỉ-lẽot} sâang \underline{Yî-lẻi-aa-sâa}; \underline{Yî-lẻi-aa-sâa} sâang \underline{Mảa-daan}; \underline{Mảa-daan} sâang \underline{Yảa-gok};
	\vs{16}\underline{Yảa-gok} sâang \underline{Mảa-lẽi-aa} ge lỏu gûng \underline{Yoek-sât}; bẽi cîng wäi Gêi-dûk ge Yë-sôu zîk hãi \underline{Mảa-lẽi-aa} sâang ge.
	
	\vs{17}Yü sõeng só sẽot, cüng \underline{Aa-baak-lâa-häam} dou \underline{Dãai-wãi} gwo zó sãp-sei dõi, cüng \underline{Dãai-wãi} heoi dou gwo \underline{Bâa-béi-lëon} ge sï hãu gwo zó sãp-sei dõi, cüng dou zó \underline{Bâa-béi-lëon} heoi dou Yë-sôu cêot sai yãu gwo zó sãp-sei dõi .
	
	\vs{18}Yë-sôu Gêi-dûk cêot sai hãi gám ge yât wüi sĩ: kẻoi aa-mâa \underline{Mảa-lẽi-aa} dông sï yỉ gîng héoi zó bei \underline{Yoek-sât}, dãan hãi hái kẻoi zũng mẽi gwo mün ge sï hãu, kẻoi zãu yân sing lëng yï dãai zó tỏu.
	\vs{19}Kẻoi lỏu gûng \underline{Yoek-sât} zok wäi yât go yĩ yän m' yũn yi dông gâai sâu-yũk kẻoi , yû-sĩ zãu nám jũ sî dái hãa yâu zó kẻoi.
	\vs{20}Dãan hãi hái kẻoi nám gán nî-dî sĩ ge sï hãu, \name{sõeng dai} ge sí-zé hái kẻoi ge mũng yãp bîn cêot yĩn, tüng kẻoi góng: \underline{Dãai-wãi} ge hãu dõi \underline{Yoek-sât}, m' hóu gêng céoi \underline{Mảa-lẽi-aa} gwo mün; kẻoi tỏu yãp bîn gó go hái sing lëng ge zái.
	\vs{21}Kẻoi wũi sâang yât go näam zái; nẻi yiu tüng go zái héi mëng giu Yë-sôu; kẻoi wũi cüng kẻoi ge yän män ge zẽoi-ok zî zûng jôeng kẻoi dẽi gáai gau cêot läi.
	\vs{22}Nî-dî sĩ hãi wãi zó yiu sïng chün \name{jú} tûng gwo sîn-zî ge háu góng ge syũt wãa:
	\vs{23}Yảu go chủ nẻoi wũi wäai yãn; kẻoi wũi sâang yât go näam zái, yï kẻoi-dẽi wũi giu kẻoi go mëng \underline{Yỉ-mảa-nõi-lẽi} (zîk hãi "\name{sõeng dai} tüng ngỏ dẽi tüng zõi" ge yi sî).
	\vs{24}\underline{Yoek-sât} séng zó zî hãu, zãu on ziu \name{sõeng dai} ge sí-zé ge fân-fũ céoi zó kẻoi lỏu pö gwo mün,
	\vs{25}dãan hãi mỏu tüng kẻoi tüng föng; zĩk dou hãu mẻi kẻoi sâang zó kẻoi ge jóeng-zí zî hãu, tüng kẻoi céoi mëng giu Yë-sôu .
	
	
	
	\ch{2}\noindent
	\vs{1}Yë-sôu hái \underline{Hêi-lẽot} zõu gán gwok-wöng ge sï-hãu hái
	\underline{Yäu-dãai} zî dẽi \underline{Bât-lẽi-häam} cêot sai; yảu géi go bok-sĩ
	cüng dûng-mĩn läi dou \underline{Yë-lõu-saat-läm} \vs{2}mãn wãa, gó-go cêot sai yiu läi zõu \underline{Yäu-taai yän} ge wöng ge
	gó-wãi hái bîn nê? Ngỏ dẽi hái dûng-mĩn ge sï-hãu tái dou
	kẻoi go lâp sêng, só yỉ dãk dâng gwo läi baai kẻoi. \vs{3}\underline{Hêi-lẽot} wöng têng dou \added{nî-dî sĩ} zî-hãu go sâm zãu hóu
  m' ôn lõk; sïng-go \underline{Yë-lõu-saat-läm} sëng ge baak-sing dôu tüng
  kẻoi yât yõeng. \vs{4}Kẻoi zĩu cäi saai chün-bõu ge zai-sî-jóeng tüng mäai
  män-gâan ge gaau-faat jûn-gâa, mãn kẻoi dẽi Gêi-dûk gau gíng
  hái bîn-bĩn cêot sai. \vs{5}Kẻoi-dẽi tüng kẻoi wãa hái \underline{Yäu-dãai} ge \underline{Bât-lẽi-häam}, yân wäi yảu sîn-zî gám sé wãa:
  
  \vs{6}\begin{poetry}
  	\underline{Bât-lẽi-häam} aa, nẻi hái \underline{Yäu-dãai} zî dẽi lẻoi-bĩn,
	  Nẻi hái \underline{Yäu-dãai} zî dẽi lẻoi-bĩn bing m' hãi zeoi m' gán yiu	gó go;
  
  Yân wãi yảu yât go túng zĩ zé wũi cüng nẻi gó dõu cêot läi,
  Kẻoi wũi túng zĩ ngỏ \underline{Yỉ-sîk-lĩ} ge zí män.
\end{poetry}
  
  \vs{7}Zî hãu \underline{Hêi-lẽot} sî dái hãa giu zó bâan bok-sĩ läi,chöeng-sai mãn kẻoi dẽi gó lâp sêng hãi hái mê sï hãu cêot yĩn ge, \vs{8}yïn zî hãu kẻoi zãu paai kẻoi-dẽi heoi \underline{Bât lẽi häam} dõu, wãa, nẻi dẽi zí-sai dî heoi wán gó-go sai lõu zái, wán dou zî hãu tüng ngỏ góng, gám ngỏ sîn hóu dôu gwo heoi baai kẻoi.  
  \vs{9}Kẻoi-dẽi têng yün gwok-wöng ge wãa zãu cêot-faat la; nî-go sï-hãu, kẻoi-dẽi hái dûng mĩn ge sï-hãu gin dou ge gó lâp sêng häang hái kẻoi-dẽi cïn bĩn, yât zĩk häang dou sai lõu zái gó bîn sîn tïng hái zîng sỏeng fông.
  \vs{10}Kẻoi-dẽi gin dou lâp sêng, sâm lẻoi dãk bĩt gôu-hing.
  \vs{11}Yïn zî hãu kẻoi-dẽi yãp dou gâan fóng gin dou go sai lõu zái tüng kẻoi aa-mâa \underline{Mảa-lẽi-aa}, yû sĩ zãu fũk dâi baai kẻoi; kẻoi dẽi dáa hôi kẻoi-dẽi daai läi ge bóu-sôeng, cïng sõeng wöng-gâm, yủ-hôeng tüng mũt-yõek zok wäi lải-mãt.
  	\vs{12}Hãu-löi yân wãi \name{Sõeng-dai} hái kẻoi-dẽi ge mũng lẻoi-bĩn gíng-gou kẻoi-dẽi m' hóu zoi fâan fâan \underline{Hêi-lẽot} gó dõu, só yỉ kẻoi-dẽi fâan kẻoi dẽi zĩ géi dẽi täu ge sï hãu häang zó lĩng ngõi yât tïu lõu.
  	
  	\vs{13}Kẻoi-dẽi záu zó zî hãu, \name{jú} ge sí-zé hái \underline{Yoek-sât} ge mũng yãp bîn cêot yĩn; tüng kẻoi wãa, héi sân lâa! Läa läa läm daai mäai go sai lõu tüng kẻoi aa-mâa heoi ôi-kãp, hái gó bîn jũ ju dáng ngỏ zoi tûng zî nẻi; \underline{Hêi-lẽot} yiu wán go sai lõu zái cêot läi saat la.
  	\vs{14}Yû-sĩ kẻoi síng zó zî hãu zãu tüng go sai lõu tüng kẻoi aa-mâa heoi zó Ôi-kãp,
  	\vs{15}hái gó bîn jũ dou \underline{Hêi-lẽot} séi zó wäi zí. Nî-dî sĩ hãi wäi zó yiu sïng chün \name{jú} tûng gwo sîn-zî góng ge wãa: Ngỏ wũi cüng Ôi-kãp giu Ngỏ ge zái cêot läi.
  	\vs{16}\underline{Hêi-lẽot} faat yĩn zĩ géi bei gó bâan bok-sĩ wán zó bãn zî hãu zãu faat saai lãan zâa, paai yän heoi \underline{Bât-lẽi-häam} tüng zâu bîn ge dẽi-kêoi, on ziu kẻoi cüng bok-sĩ têng fâan läi ge sï gâan jôeng lỏeng seoi yỉ hãa ge sai lõu chün-bõu saat saai; 
  	\vs{17}gám yõeng zãu sïng chün zó sîn-zî \underline{Yë-lẽi-mải} góng ge syũt-wãa:
  	\vs{18}\begin{poetry}
  		Cüng \underline{Lâa-mảa} sïng chün löi yât báa sêng,
  		Bêi tung, câu yâp, tüng ôi dõu;
  		
  		\underline{Lâa-hit} wãi kẻoi ge zái-nẻoi yï hûk;
  		Möu yän hó yỉ ôn wai dou kẻoi, yân wäi kẻoi ge zái-nẻoi chün-bõu dou m' hái dõu la.
  	\end{poetry}
  	
  	\vs{19}\underline{Hêi-lẽot} séi zó zî hãu, \name{sõeng-dai} ge sí-zé hái \underline{Yoek-sât} ge mũng yãp bîn cêot yĩn; 
  	\vs{20}tüng kẻoi wãa, héi sân lâa! Daai mäai go sai lõu tüng kẻoi aa-mâa fâan \underline{Yỉ-sîk-lĩt} lâa, yiu saat nẻi go zái ge yän yỉ gîng séi zó la.
  	\vs{21}Yû-sĩ kẻoi zãu héi sân daai go sai lõu tüng kẻoi aa-mâa fâan zó \underline{Yỉ-sîk-lĩt};
  	\vs{22}dãan hãi kẻoi têng dou \underline{Aa-hêi-läu-sî} gai sïng zó kẻoi lỏu dãu \underline{Hêi-lẽot} go wãi zõu zó \underline{Yäu-dãai} zî dẽi ge gwok wöng ge sï hãu, hôi cí gok dâk gêng yï m' gám fâan heoi; yï-cé \name{sõeng-dai} hái kẻoi ge mũng yãp bîn dôu gíng gou gwo kẻoi, yû-sĩ kẻoi zãu jun sân heoi zó \underline{Gâa-lẽi-lẽi};
  	\vs{23}Kẻoi hái yât go giu \underline{Näa-saat-lãak} ge sïng dõu jũ dâi; gám yõeng zãu sïng chün zó sîn-zî góng ge ``Kẻoi wũi bei yän cîng wäi hãi \underline{Näa-saat-lãak} yän'' nî gĩn sĩ.
  	
  	
  	

\end{scripture}
\end{document}
